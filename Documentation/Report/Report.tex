\documentclass[letterpaper]{article}
    \usepackage{amsmath}
    \usepackage{tikz}
    \usepackage{epigraph}
    \usepackage{lipsum}
    \usepackage{glossaries}
    \usepackage{graphicx}
    \usepackage{imakeidx}
    \usepackage{hyperref}
    \graphicspath{{Images/}}
    \setcounter{tocdepth}{2}
    \makeglossaries
    \makeindex[columns=2, title=Alphabetical Index]
    \renewcommand\epigraphflush{flushright}
    \renewcommand\epigraphsize{\normalsize}
    \setlength\epigraphwidth{0.7\textwidth}

    \definecolor{titlepagecolor}{cmyk}{1,.60,0,.40}

    \DeclareFixedFont{\titlefont}{T1}{ppl}{b}{it}{0.5in}

    \makeatletter
    \def\printauthor{%
        {\large \@author}}
    \makeatother
    \author{
        Sam Miller \\
       }

    % The following code is borrowed from:  https://tex.stackexchange.com/a/86310/10898

    \newcommand\titlepagedecoration{%
    \begin{tikzpicture}[remember picture,overlay,shorten >= -10pt]

    \coordinate (aux1) at ([yshift=-15pt]current page.north east);
    \coordinate (aux2) at ([yshift=-410pt]current page.north east);
    \coordinate (aux3) at ([xshift=-4.5cm]current page.north east);
    \coordinate (aux4) at ([yshift=-150pt]current page.north east);

    \begin{scope}[titlepagecolor!40,line width=12pt,rounded corners=12pt]
    \draw
      (aux1) -- coordinate (a)
      ++(225:5) --
      ++(-45:5.1) coordinate (b);
    \draw[shorten <= -10pt]
      (aux3) --
      (a) --
      (aux1);
    \draw[opacity=0.6,titlepagecolor,shorten <= -10pt]
      (b) --
      ++(225:2.2) --
      ++(-45:2.2);
    \end{scope}
    \draw[titlepagecolor,line width=8pt,rounded corners=8pt,shorten <= -10pt]
      (aux4) --
      ++(225:0.8) --
      ++(-45:0.8);
    \begin{scope}[titlepagecolor!70,line width=6pt,rounded corners=8pt]
    \draw[shorten <= -10pt]
      (aux2) --
      ++(225:3) coordinate[pos=0.45] (c) --
      ++(-45:3.1);
    \draw
      (aux2) --
      (c) --
      ++(135:2.5) --
      ++(45:2.5) --
      ++(-45:2.5) coordinate[pos=0.3] (d);
    \draw
      (d) -- +(45:1);
    \end{scope}
    \end{tikzpicture}%
    }




\begin{document}
\begin{titlepage}

	\noindent
	\titlefont CS4610 \\Project Management Plan\par
	\epigraph{Project Report for Mumba\\ November 18, 2018}%
	{\textit{}\\ \textsc{}}
	\null\vfill
	\vspace*{1cm}
	\noindent
	\hfill
	\begin{minipage}{0.35\linewidth}
		\begin{flushright}
			\printauthor
		\end{flushright}
	\end{minipage}
	%
	\begin{minipage}{0.02\linewidth}
		\rule{1pt}{125pt}
	\end{minipage}
	\titlepagedecoration
\end{titlepage}
\tableofcontents
\pagebreak

\section{Introduction}

Mumba (https://mumba.azurewebsites.com) provides a simple no frills product management solution for small solo projects like those seen in university classes. My website is open to all and I hope that my project can be of use to myself and others in managing their projects and tasks. Anyone can register an account and use my application. Once the user has logged in they can boards which are collections of lists. They can then open the board and create tasks/cards in different lists to keep track of steps to complete a project.

My Project is implemented as a full stack C\# ASP.Net Core application, I used the MVC pattern to dynamically manage the views returned to the users of my application. The buttons and interactions a user has with the website are controlled by one of the Four controller classes which will be detailed in the Description of Components section. The SQL server behind my web app is dynamically called to bring forth only the correct data that belongs only to the user who requested them.

\section{Related Works}

The core functionality of Mumba is inspired by project boards (commonly known as  a kanban board) is commonly seen in Atlassian's Trello.com. Mumba is much simpler, there are not as many features as included in Trello, for example in Mumba's current state each board has 3 lists and only 3 lists, Trello dynamically allows creation and deletion of lists.

\section{System Architectural Design}



\subsection{Chosen System Architecure}

\section{Detailed Description of Components}

\section{Screenshots}

\section{Documentation}

\section{Conclusion}



\end{document}
